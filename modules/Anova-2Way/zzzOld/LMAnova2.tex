%===============================================================================
% swv2PDF("LMAnova2.Rnw",createScript=FALSE)
%===============================================================================
% Overall Sweave and R options
%-------------------------------------------------------------------------------


%===============================================================================

\documentclass{article}
\input{c:/aaaWork/zGnrlLatex/GnrlPreamble}
\usepackage{Sweave}
\begin{document}

\title{Two-Way ANOVA Handout}
\date{}  % makes date blank -- title will only have the title above then
\maketitle
\vspace{-36pt}
\section{Initialization} \label{sect:Inits}
\vspace{-18pt}
\begin{Schunk}
\begin{Sinput}
> library(NCStats)
> library(gdata)     # for aggregate.table()
> library(multcomp)  # for glht()
\end{Sinput}
\end{Schunk}

\vspace{-24pt}
\section{Bacteria Example}
\vspace{-18pt}
\begin{Schunk}
\begin{Sinput}
> bact <- read.table("Bacteria.txt",header=TRUE)
> str(bact)
\end{Sinput}
\begin{Soutput}
'data.frame':	30 obs. of  3 variables:
 $ temp : int  27 27 27 27 27 35 35 35 35 35 ...
 $ conc : num  0.6 0.8 1 1.2 1.4 0.6 0.8 1 1.2 1.4 ...
 $ cells: int  55 120 186 260 151 82 166 179 223 178 ...
\end{Soutput}
\begin{Sinput}
> bact$ftemp <- factor(bact$temp)
> bact$fconc <- factor(bact$conc)
> str(bact)
\end{Sinput}
\begin{Soutput}
'data.frame':	30 obs. of  5 variables:
 $ temp : int  27 27 27 27 27 35 35 35 35 35 ...
 $ conc : num  0.6 0.8 1 1.2 1.4 0.6 0.8 1 1.2 1.4 ...
 $ cells: int  55 120 186 260 151 82 166 179 223 178 ...
 $ ftemp: Factor w/ 3 levels "27","35","43": 1 1 1 1 1 2 2 2 2 2 ...
 $ fconc: Factor w/ 5 levels "0.6","0.8","1",..: 1 2 3 4 5 1 2 3 4 5 ...
\end{Soutput}
\end{Schunk}

Explore sample size per group, univariate statistics, statistics for each group, and table of group means,
\begin{Schunk}
\begin{Sinput}
> with(bact,table(ftemp,fconc))
\end{Sinput}
\begin{Soutput}
     fconc
ftemp 0.6 0.8 1 1.2 1.4
   27   2   2 2   2   2
   35   2   2 2   2   2
   43   2   2 2   2   2
\end{Soutput}
\begin{Sinput}
> Summarize(cells~ftemp,data=bact)
\end{Sinput}
\begin{Soutput}
  ftemp  n  Mean St. Dev. Min. 1st Qu. Median 3rd Qu. Max.
1    27 10 153.1 69.83544   55   113.2  141.5   177.2  274
2    35 10 169.3 51.14044   82   157.0  172.0   209.0  236
3    43 10 161.5 33.95176  115   138.0  149.5   185.2  228
\end{Soutput}
\begin{Sinput}
> Summarize(cells~fconc,data=bact)
\end{Sinput}
\begin{Soutput}
  fconc n     Mean St. Dev. Min. 1st Qu. Median 3rd Qu. Max.
1   0.6 6 108.1667 38.77843   55    85.0  104.5   141.2  153
2   0.8 6 144.1667 34.00245   91   126.5  151.0   163.5  186
3     1 6 154.8333 23.69318  133   135.2  148.0   174.2  186
4   1.2 6 235.0000 29.85297  189   224.2  232.0   254.0  274
5   1.4 6 164.3333 37.30773  111   145.8  164.5   181.8  219
\end{Soutput}
\begin{Sinput}
> Summarize(cells~ftemp*fconc,numdigs=2,data=bact)
\end{Sinput}
\begin{Soutput}
   ftemp fconc n  Mean St. Dev. Min. 1st Qu. Median 3rd Qu. Max.
1     27   0.6 2 102.5    67.18   55   78.75  102.5   126.2  150
2     35   0.6 2  88.0     8.49   82   85.00   88.0    91.0   94
3     43   0.6 2 134.0    26.87  115  124.50  134.0   143.5  153
4     27   0.8 2 105.5    20.51   91   98.25  105.5   112.8  120
5     35   0.8 2 161.0     7.07  156  158.50  161.0   163.5  166
6     43   0.8 2 166.0    28.28  146  156.00  166.0   176.0  186
7     27     1 2 159.5    37.48  133  146.20  159.5   172.8  186
8     35     1 2 169.5    13.44  160  164.80  169.5   174.2  179
9     43     1 2 135.5     0.71  135  135.20  135.5   135.8  136
10    27   1.2 2 267.0     9.90  260  263.50  267.0   270.5  274
11    35   1.2 2 229.5     9.19  223  226.20  229.5   232.8  236
12    43   1.2 2 208.5    27.58  189  198.80  208.5   218.2  228
13    27   1.4 2 131.0    28.28  111  121.00  131.0   141.0  151
14    35   1.4 2 198.5    28.99  178  188.20  198.5   208.8  219
15    43   1.4 2 163.5    27.58  144  153.80  163.5   173.2  183
\end{Soutput}
\begin{Sinput}
> with(bact,aggregate.table(cells,ftemp,fconc,FUN=mean))
\end{Sinput}
\begin{Soutput}
     0.6   0.8     1   1.2   1.4
27 102.5 105.5 159.5 267.0 131.0
35  88.0 161.0 169.5 229.5 198.5
43 134.0 166.0 135.5 208.5 163.5
\end{Soutput}
\end{Schunk}

Fit the model and get the ANOVA results
\begin{Schunk}
\begin{Sinput}
> lm1 <- lm(cells~ftemp*fconc,data=bact)
> anova(lm1)
\end{Sinput}
\begin{Soutput}
Analysis of Variance Table

Response: cells
            Df Sum Sq Mean Sq F value    Pr(>F)
ftemp        2   1313   656.4  0.8557   0.44473
fconc        4  51596 12899.1 16.8154 2.041e-05
ftemp:fconc  8  14703  1837.8  2.3958   0.06886
Residuals   15  11507   767.1                  
\end{Soutput}
\end{Schunk}


Examine the interaction plots

\begin{Schunk}
\begin{Sinput}
> fitPlot(lm1)
> fitPlot(lm1,interval=FALSE,change.order=TRUE,xlab="Concentration (%)",
  ylab="Mean Number of Cells",legend="topleft")
\end{Sinput}
\end{Schunk}
\includegraphics[width=6in]{Figs/lma2-BactIntPlot.PDF}

Examine the main effects plots
\begin{Schunk}
\begin{Sinput}
> fitPlot(lm1,which="ftemp",xlab="Temperature (C)",
  ylab="Mean Number of Cells",ylim=c(60,270))
> fitPlot(lm1,which="fconc",xlab="Concentration (%)",
  ylab="Mean Number of Cells",ylim=c(60,270))
\end{Sinput}
\end{Schunk}
\includegraphics[width=6in]{Figs/lma2-BactMainPlot1.PDF}

\newpage
Examine Tukey's HSD results for concentration main effect and construct a main effects plot with significance letters.

\begin{Schunk}
\begin{Sinput}
> bact.mc1 <- glht(lm1,mcp(fconc="Tukey"))
> summary(bact.mc1)
\end{Sinput}
\begin{Soutput}
	 Simultaneous Tests for General Linear Hypotheses

Multiple Comparisons of Means: Tukey Contrasts


Fit: lm(formula = cells ~ ftemp * fconc, data = bact)

Linear Hypotheses:
               Estimate Std. Error t value Pr(>|t|)
0.8 - 0.6 == 0      3.0       27.7   0.108 0.999965
1 - 0.6 == 0       57.0       27.7   2.058 0.287200
1.2 - 0.6 == 0    164.5       27.7   5.939 0.000196
1.4 - 0.6 == 0     28.5       27.7   1.029 0.838190
1 - 0.8 == 0       54.0       27.7   1.950 0.334998
1.2 - 0.8 == 0    161.5       27.7   5.831 0.000275
1.4 - 0.8 == 0     25.5       27.7   0.921 0.884508
1.2 - 1 == 0      107.5       27.7   3.881 0.010932
1.4 - 1 == 0      -28.5       27.7  -1.029 0.838184
1.4 - 1.2 == 0   -136.0       27.7  -4.910 0.001464
(Adjusted p values reported -- single-step method)
\end{Soutput}
\begin{Sinput}
> fitPlot(lm1,which="fconc",xlab="Concentration (%)",ylab="Mean Number of Cells",main="")
> addSigLetters(lm1,which="fconc",lets=c("a","ab","ab","c","b"),pos=c(2,2,4,2,4))
\end{Sinput}
\end{Schunk}
\includegraphics[width=3in]{Figs/lma2-BactMainPlot2.PDF}


\newpage
\section{Soil Phosphorous Example}
You must change the directory to where the following file is located.
\begin{Schunk}
\begin{Sinput}
> sp <- read.table("SoilPhosphorous.txt",header=TRUE)
> str(sp)
\end{Sinput}
\begin{Soutput}
'data.frame':	24 obs. of  3 variables:
 $ soil: Factor w/ 2 levels "sandstone","shale": 2 2 2 2 2 2 2 2 2 2 ...
 $ topo: Factor w/ 4 levels "hilltop","north",..: 4 4 4 2 2 2 3 3 3 1 ...
 $ phos: int  98 172 185 78 77 100 117 54 96 83 ...
\end{Soutput}
\begin{Sinput}
> lm1 <- lm(phos~soil*topo,data=sp)
> leveneTest(lm1)
\end{Sinput}
\begin{Soutput}
Levene's Test for Homogeneity of Variance (center = median)
      Df F value Pr(>F)
group  7  0.3741 0.9043
      16               
\end{Soutput}
\end{Schunk}
\begin{Schunk}
\begin{Sinput}
> residPlot(lm1)
\end{Sinput}
\end{Schunk}
\includegraphics[width=3in]{Figs/lma2-SoilResidPlot.PDF}

\begin{Schunk}
\begin{Sinput}
> adTest(lm1$residuals)
\end{Sinput}
\begin{Soutput}
	Anderson-Darling normality test

data:  lm1$residuals 
A = 0.2126, p-value = 0.8351
\end{Soutput}
\begin{Sinput}
> outlierTest(lm1)
\end{Sinput}
\begin{Soutput}
No Studentized residuals with Bonferonni p < 0.05
Largest |rstudent|:
   rstudent unadjusted p-value Bonferonni p
1 -2.824098           0.012821      0.30769
\end{Soutput}
\begin{Sinput}
> anova(lm1)
\end{Sinput}
\begin{Soutput}
Analysis of Variance Table

Response: phos
          Df  Sum Sq Mean Sq F value    Pr(>F)
soil       1 17876.0 17876.0 22.9818 0.0001988
topo       3  9693.8  3231.3  4.1542 0.0235128
soil:topo  3 11390.8  3796.9  4.8814 0.0134826
Residuals 16 12445.3   777.8                  
\end{Soutput}
\end{Schunk}

When a two-way ANOVA model has a significant interaction term then multiple comparisons must be computed between each group that can be identified as combinations of the two factors.  Unfortunately, this is not a straightforward calculation with the \R{glht()} function.  However, the calculation can be made relatively easily by creating a single factor that consists of the combinations of the two original factors, fitting a one-way ANOVA model to this new single factor, and then submitting this result to the \R{glht()} function.  This process is illustrated below.

\begin{Schunk}
\begin{Sinput}
> sp$comb <- sp$soil:sp$topo
> view(sp)
\end{Sinput}
\begin{Soutput}
        soil    topo phos             comb
7      shale   south  117      shale:south
11     shale hilltop   12    shale:hilltop
13 sandstone  valley   19 sandstone:valley
16 sandstone   north   27  sandstone:north
17 sandstone   north   49  sandstone:north
21 sandstone   south   72  sandstone:south
\end{Soutput}
\begin{Sinput}
> lm1a <- lm(phos~comb,data=sp)
> anova(lm1a)
\end{Sinput}
\begin{Soutput}
Analysis of Variance Table

Response: phos
          Df Sum Sq Mean Sq F value    Pr(>F)
comb       7  38961  5565.8  7.1555 0.0005729
Residuals 16  12445   777.8                  
\end{Soutput}
\begin{Sinput}
> spint.mc <- glht(lm1a, mcp(comb="Tukey"))
> summary(spint.mc)
\end{Sinput}
\begin{Soutput}
	 Simultaneous Tests for General Linear Hypotheses

Multiple Comparisons of Means: Tukey Contrasts


Fit: lm(formula = phos ~ comb, data = sp)

Linear Hypotheses:
                                          Estimate Std. Error t value Pr(>|t|)
sandstone:north - sandstone:hilltop == 0     1.667     22.772   0.073  1.00000
sandstone:south - sandstone:hilltop == 0    19.333     22.772   0.849  0.98682
sandstone:valley - sandstone:hilltop == 0   -4.000     22.772  -0.176  1.00000
shale:hilltop - sandstone:hilltop == 0       4.667     22.772   0.205  1.00000
shale:north - sandstone:hilltop == 0        53.333     22.772   2.342  0.33035
shale:south - sandstone:hilltop == 0        57.333     22.772   2.518  0.25579
shale:valley - sandstone:hilltop == 0      120.000     22.772   5.270  0.00145
sandstone:south - sandstone:north == 0      17.667     22.772   0.776  0.99218
sandstone:valley - sandstone:north == 0     -5.667     22.772  -0.249  1.00000
shale:hilltop - sandstone:north == 0         3.000     22.772   0.132  1.00000
shale:north - sandstone:north == 0          51.667     22.772   2.269  0.36548
shale:south - sandstone:north == 0          55.667     22.772   2.445  0.28527
shale:valley - sandstone:north == 0        118.333     22.772   5.196  0.00180
sandstone:valley - sandstone:south == 0    -23.333     22.772  -1.025  0.96345
shale:hilltop - sandstone:south == 0       -14.667     22.772  -0.644  0.99747
shale:north - sandstone:south == 0          34.000     22.772   1.493  0.80043
shale:south - sandstone:south == 0          38.000     22.772   1.669  0.70534
shale:valley - sandstone:south == 0        100.667     22.772   4.421  0.00788
shale:hilltop - sandstone:valley == 0        8.667     22.772   0.381  0.99992
shale:north - sandstone:valley == 0         57.333     22.772   2.518  0.25516
shale:south - sandstone:valley == 0         61.333     22.772   2.693  0.19393
shale:valley - sandstone:valley == 0       124.000     22.772   5.445  0.00109
shale:north - shale:hilltop == 0            48.667     22.772   2.137  0.43376
shale:south - shale:hilltop == 0            52.667     22.772   2.313  0.34385
shale:valley - shale:hilltop == 0          115.333     22.772   5.065  0.00215
shale:south - shale:north == 0               4.000     22.772   0.176  1.00000
shale:valley - shale:north == 0             66.667     22.772   2.928  0.13113
shale:valley - shale:south == 0             62.667     22.772   2.752  0.17669
(Adjusted p values reported -- single-step method)
\end{Soutput}
\end{Schunk}

\begin{Schunk}
\begin{Sinput}
> fitPlot(lm1,change.order=TRUE,interval=FALSE,main="",ylab="Mean Phosphorous Level",
  xlab="Topographic Location",legend="topleft")
> addSigLetters(lm1,change.order=TRUE,lets=c("a","a","a","ab","a","ab","a","b"),
  pos=c(1,3,1,3,1,1,3,1))
\end{Sinput}
\end{Schunk}
\includegraphics[width=3in]{Figs/lma2-SoilFLP1.PDF}


The following code can be used to isolate the multiple comparisions that have a p-value less than 0.05.
\begin{Schunk}
\begin{Sinput}
> spmc <- summary(spint.mc)
> names(spmc$test$coefficients)[spmc$test$pvalues<0.05]
\end{Sinput}
\begin{Soutput}
[1] "shale:valley - sandstone:hilltop" "shale:valley - sandstone:north"  
[3] "shale:valley - sandstone:south"   "shale:valley - sandstone:valley" 
[5] "shale:valley - shale:hilltop"    
\end{Soutput}
\end{Schunk}
\end{document}
