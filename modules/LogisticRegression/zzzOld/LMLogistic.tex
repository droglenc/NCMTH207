% setwd("C://aaaWork//Class Materials//MTH207//Lecture/Handouts//")
% Sweave("LMLogistic.Rnw",stylepath=TRUE,syntax="SweaveSyntaxNoweb")

\documentclass[a4paper]{article}
\input{c:/aaaWork/zGnrlLatex/GnrlPreamble}

\usepackage{C:/Apps/R-2.11.1/share/texmf/Sweave}
\begin{document}


\title{Logistic Regression Handout}
\date{}  % makes date blank -- title will only have the title above then
\maketitle
\vspace{-72pt}


\section{Initialization} \label{sect:Inits}
\begin{Schunk}
\begin{Sinput}
> library(NCStats)
\end{Sinput}
\end{Schunk}

\section{Bat Subspecies Example}
You must change the directory to where the following file is located.  In addition, I changed the canine measurements to mm (from cm) for ease of explanation later on.
\begin{Schunk}
\begin{Sinput}
> bat <- read.table("BatMorph.txt",head=TRUE)
> str(bat)
\end{Sinput}
\begin{Soutput}
'data.frame':	118 obs. of  7 variables:
 $ subsp      : Factor w/ 2 levels "cinereus","semotus": 2 2 2 2 2 2 2 2 2 2 ...
 $ bodymass   : num  19.5 16.2 17 16.5 14.3 ...
 $ skulllength: num  1.6 1.55 1.56 1.56 1.53 ...
 $ canine     : num  0.326 0.308 0.291 0.287 0.301 0.305 0.277 0.313 0.289 0.293 ...
 $ coronoid   : num  0.303 0.282 0.292 0.303 0.279 0.284 0.286 0.281 0.278 0.28 ...
 $ wingspan   : num  0.358 0.358 0.359 0.353 0.351 0.361 0.351 0.363 0.34 0.365 ...
 $ hab        : Factor w/ 3 levels "A","B","C": 1 1 1 1 1 1 1 1 2 2 ...
\end{Soutput}
\begin{Sinput}
> bat$canine <- bat$canine*10
\end{Sinput}
\end{Schunk}
\begin{Schunk}
\begin{Sinput}
> cdplot(subsp~canine,data=bat,ylevels=2:1)
\end{Sinput}
\end{Schunk}
\includegraphics[width=3in]{Figs/lmlog-BatCDPlot.PDF}

\begin{Schunk}
\begin{Sinput}
> glm1 <- glm(subsp~canine,data=bat,family=binomial)
> logregPlot(glm1,p.ints=15,xlab="Canine Height",ylab="Subspecies Code")
\end{Sinput}
\end{Schunk}
\includegraphics[width=3in]{Figs/lmlog-BatFitPlot.PDF}

\begin{Schunk}
\begin{Sinput}
> coef(glm1)
\end{Sinput}
\begin{Soutput}
(Intercept)      canine 
   35.51574   -11.11193 
\end{Soutput}
\begin{Sinput}
> confint(glm1)
\end{Sinput}
\begin{Soutput}
                2.5 %   97.5 %
(Intercept)  24.21685 49.66132
canine      -15.52430 -7.58941
\end{Soutput}
\begin{Sinput}
> predict(glm1,data.frame(canine=c(3,4)))
\end{Sinput}
\begin{Soutput}
        1         2 
 2.179940 -8.931994 
\end{Soutput}
\begin{Sinput}
> -8.931994-2.179940
\end{Sinput}
\begin{Soutput}
[1] -11.11193
\end{Soutput}
\begin{Sinput}
> exp(coef(glm1))
\end{Sinput}
\begin{Soutput}
 (Intercept)       canine 
2.656377e+15 1.493306e-05 
\end{Soutput}
\begin{Sinput}
> exp(predict(glm1,data.frame(canine=c(3,4))))
\end{Sinput}
\begin{Soutput}
           1            2 
8.8457728416 0.0001320944 
\end{Soutput}
\begin{Sinput}
> 0.0001320944/8.8457728416
\end{Sinput}
\begin{Soutput}
[1] 1.493305e-05
\end{Soutput}
\begin{Sinput}
> summary(glm1)
\end{Sinput}
\begin{Soutput}
Call:
glm(formula = subsp ~ canine, family = binomial, data = bat)

Deviance Residuals: 
    Min       1Q   Median       3Q      Max  
-1.9483  -0.6384  -0.1377   0.5923   2.2658  

Coefficients:
            Estimate Std. Error z value Pr(>|z|)
(Intercept)   35.516      6.428   5.525 3.29e-08
canine       -11.112      2.005  -5.543 2.97e-08

(Dispersion parameter for binomial family taken to be 1)

    Null deviance: 163.040  on 117  degrees of freedom
Residual deviance:  97.178  on 116  degrees of freedom
AIC: 101.18

Number of Fisher Scoring iterations: 5
\end{Soutput}
\begin{Sinput}
> predict(glm1,data.frame(canine=c(3,3.4)))
\end{Sinput}
\begin{Soutput}
        1         2 
 2.179940 -2.264834 
\end{Soutput}
\begin{Sinput}
> predict(glm1,data.frame(canine=c(3,3.4)),type="response")
\end{Sinput}
\begin{Soutput}
        1         2 
0.8984336 0.0940776 
\end{Soutput}
\end{Schunk}

\end{document}
